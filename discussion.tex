Using sequence graphs as reference structures
\subsection{Complexity}
Indexing a sequence graph has been shown to be inherently difficult~\cite{indexgraphs}. It is therefore no surprise that \emph{vg}, which is the graph mapper that shows the most promising results suffers from long run times and high memory consumption. 

\subsection{Kmer Sinks}
The combinatorial growth in subsequences from variant density not only leads to problems with computational efficiency. It also leads to possibilities for false positives.
As is briefly discussed in Paper 3, areas with a high density of variants, can represent a large number of possible sequences.
This can lead to such areas matching read sequences which does not originate from that area. 
Such mismappings is a problem for the mapping accuracy in general and the bias it introduces can be a problem for downstream analysis. 

\subsection{Pseudo-linearity}



%%% Local Variables:
%%% mode: latex
%%% TeX-master: "main"
%%% End:
