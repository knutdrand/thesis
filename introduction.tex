The broad topic of this thesis graph based reference genomes, and specifically of mapping and peak calling on such references.
The introduction here is meant to provide an introduction to the topic as well as a coherent brief of the content of the provided papers.
As is common in bioinformatics, full understanding of the field of graph based references requires at least a rudimentary understanding of the underlying biology, mathematics and informatics. An brief introduction to the specific subtopics 

The first section will be an introduction to two main biological topics, reference genomes and genomic variation,
in order to familiarize the reader with topics in addition to establish a perspective that establishes the benefits of graph based references.
Section 2 introduces the concept of sequence graphs and their interpretation in different settings.
Section 3 describes the mathematical formalism used in this thesis as well as the data structures embodying this formalism.
These sections might be persived as too rigorous and opaque for a brief introduction, but this highlights one of the main drawbacks of graphical models, namely their complexity.
% These concepts were introduced in Paper 1: \emph{Coordinates and intervals in Graph Based Reference Genomes}, and further developed in the subsequent papers.
Section 4 and 5 gets to the main focus of this thesis, covering the topics of read mapping and peak calling on graph based reference genomes.


%%% Local Variables:
%%% mode: latex
%%% TeX-master: "main"
%%% End:
