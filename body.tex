In this thesis four papers are presented that each relates to a different part of how reference genomes are used in bioinformatics.
Read in the correct order, these can be seen to follow a common bioinformatics pipeline from start to end: Mapping sequenced reads to a reference genome (Paper III), using mapped reads to predict protein binding regions (Paper II), representing these binding regions as intervals on the reference (Paper I), and using these intervals in a colocalization analysis (Paper IV).

What separates this example pipeline from a normal bioinformatics pipeline is that the first three steps are performed using a graph based reference genome as opposed to a linear reference.
The implications of using a graph based reference is the main focus of this thesis.

In order to give context to these papers and the concept of graph based reference structures, some background is provided in Chapter~\ref{sec:background}.
Section~\ref{sec:biology} introduces important concepts from molecular biology, the subject matter which the methodologies used in this thesis are meant to elucidate.
DNA-sequencing, the measuring process providing the raw data (sequenced reads) for the analyses in this thesis, is briefly explained in Section~\ref{sec:sequencing}.
Section~\ref{sec:refgenomes} outlines how reference genomes are used to help make sense of this raw data.
% Section~\ref{sec:sequencing} introduces methods in DNA-sequencing, which is the process where biological molecules are.
% The central concept of reference genomes is explained in Section~\ref{sec:refgenomes}.

Section~\ref{sec:notation} provides the mathematical framework used in the last, and most technical, part of the background which gives brief explanations of the relevant algorithms used to make inferences, using reference genomes, based on sequenced reads.
The related concepts of sequence alignment (Section~\ref{sec:alignment}) and mapping (Section~\ref{sec:mapping}) is followed by peak calling (Section~\ref{sec:peakcalling}).

In total this background is meant to elucidate the role of reference genomes in bioinformatics, and provide a background for what a change to the use of graph based reference genomes entails.

Finally, Chapter~\ref{sec:summary} provides a brief summary of each paper provided, before the final discussion and conclusion of the thesis (Chapter~\ref{sec:discussion}).

%
% 
% 
% e adapted to make use of a more detailed graph based reference genome, and the consequences such a change involves.
% 
% 
% 
% s to answer biological questions.
% ogether covering reference genomes' use from raw data to biological results.
% 
% 
% 
% in a genome.
% 
% ferent results.
% 
% ons.
% 
% 
% .
% 
% 
% ream analysis to deduce biochemical properties of DNA,
% e genome,
% biochemical processes.
% ence genomes.
% 
%  of reference genomes in bioinformatics.
% ref{sec:sequencing} explains the process of DNA-sequencing which is how the observed data used in this thesis was obtained.
% ctions (\ref{sec:alignment}, \ref{sec:mapping}) on alignment and mapping which introduces some technical aspects of one of the primary functions of reference genomes.
% heme of Paper II. 
% ed, as well as a motivation for using graph based reference genomes.
% 
% 
% erences.
% he provided papers.
% ary understanding of the underlying biology, mathematics and informatics. An brief introduction to the specific subtopics 
% 
% 
%  graph based references.
% 
% lism.
% ain drawbacks of graphical models, namely their complexity.
% rther developed in the subsequent papers.
% ased reference genomes.


%%% Local Variables:
%%% mode: latex
%%% TeX-master: "main"
%%% End:

Since it can be assumed that the genomic DNA-sequences from individuals of the same population is highly similar, it is possible to use one DNA-sequence as a \emph{reference genome} for the population.
Such reference genomes serves two main purposes.
They make the deduction of the DNA-sequence from a sample a simpler problem through mapping, and they make it possible to represent the outcome of a range of biological experiments as intervals or coordinates on the coordinate system induced by the reference genome. The process of mapping is covered in section~\ref{sec:mapping}, while the representation of different experiment outputs are illustrated in figure~\ref{fig:refpos}.

\begin{figure}
  \includegraphics{figures/refgenomes}
  \label{fig:refpos}
  \caption{Figure showing the locations of several genomic elements on the same coordinate system.
    Locations of genes (red), transcription factor binding sites (blue), and methylation sites (green) from different sequencing experiments can all be compared and analyzed together.}
\end{figure}


% \subsection{Experiments}
% 
% Since the output from NGS-experiments are large amounts of short reads, reference genomes have been an invaluable resource.
% Most importantly they simplify the analysis of short read data by mapping them 
% 
% 
% Reference genomes have become a vital part of bioinformatic analysis.
% 
% \subsection{As a }
% The concept of a reference genome is that the genomic sequence between two individuals can be highly similar. Thus knowing the genomic sequence of one individual.
% 
% \subsection{Data Integration}
% 
% 
% Genomic variation is one of the fundamental aspects of biology.
% Difference in the DNA-sequence between two individuals can lead to a change in the translated RNA sequence and further in the sequence, and thereby the form and function, of the expressed protein.
% Or it can lead to less drastic changes such as changes in the RNA-structure or the shape of the DNA molecule itself.
% In this way genomic variation can determine differences between specimens of the same species and also differences between the species themselves.
% 
% Within a species, the genomic variation between individuals are often limited by evolutionary conservative pressure, 
% meaning that the difference in DNA-sequence between viable specimens of the same species are often small and does not lead to big changes in neither the structure of the DNA or the translated proteins.
% This limitation has made  possible the use of \emph{reference genome}s for a species, a DNA-sequence though to represent the generic sequence for that species, where individual variation from the reference sequence are though to be small.
% 
% Reference genomes have helped make sense of sequencing data, that have been dominated by large numbers of small sequence fragments.
% Without a good reference genome, one would need to fit all those sequence fragments together like a puzzle, in a process called \emph{assembly}.
% Using a reference genome, one can instead find the best sequence match for each read in the reference sequence, and thus find both where each read belongs in the genome, and also how the sequence differs from the reference.
% Finding the position of each read can give context to them, since the location of biologically important parts of the genome can be represented on the genome.
% For instance, if the sequencing of an individual maps to a location within a the known location of a protein coding gene and differs from the reference sequence, it's possible that that individual has a genomic variant that alters the function of that protein. 
% 
% Since this process of \emph{mapping} sequence reads to the reference is such a fundamental step in many biological analyses, the quality of the reference sequence has been of high importance.
% Thus, since the dawn of human genome sequencing in \TODO{year}, the Genome Reference Consortium has released \TODO{n} versions of the human reference genome alone.
% The latest version of the human reference genome (GRCh38), highlights some shortcomings of representing the reference of a species as a single sequence.
% GRCh38 includes, in addition to the traditional linear reference sequence, a set of alternative sequences from areas of the genome where there are known variations of the DNA-sequences which makes it problematic to map reads from those areas to the reference sequences.
% Secondly the new version included changes that disrupted the coordiante system of the reference.
% This has lead to a backward incompatibility that has prevented a widespread adoption of the new reference.
% 
% \subsection{Mapping Bias}
% Mapping sequencing reads to the reference entails finding subsequences in the reference sequence that are highly similar to the sequenced reads.
% The match can be inexact due to either the actual DNA-sequence being different or sequencing errors that can substitute one nucleotide with another.
% It is necessary to set a limit to how dissimilar the reference subsequence can be in order to produce a match due to computational complexity and that allowing a high level of divergence can lead to a large number of matches.
% For instance, \emph{BWA-mem} by default requires a shared subsequence of at least 19 bp in order to produce a match.
% This means that reads from regions with much variation from the reference can be unmappable using standard mapping software, and be susceptible to small amount of sequenceing error yielding them unmappable.
% This means that the resemblence of the sample to the reference genome will lead to better mapping quality, and that some regions of the genome will have a lower mapping rate than others. 
% 
% 
% \subsection{Geometry of the reference genome}
% As well as being an indexable lookup table for sequencing reads, a reference genome provides a coordinate system and a geometry for sequence data that allows us to look at different sequence elements in conjunction. 
% Most important is the analysis of overlap and distances between subsequences.
% For instance, the location of a  potential transcription factor binding site, predicted from a ChIP-seq experiment, can be compared to the positions of known genes, predicted from amongst others RNA-seq experiments, to determine which gene is most likely regulated by the binding of a TF to the binding site.
% The distances between subsequences is also used in some mapping tools themselves, as a way to evaluate the match of a read to a reference subsequence.
% For instance Minimap2~\cite{minimap2} uses the relative positioning of kmer matches to the reference to find which chain of kmer-anchors match the read sequence best.
% The distance between two subsequences on the reference is invariant to SNPs, since they do not affect distances.
% However indels and especially structural variants affect the distances between subsequences, and can thus change the outcome of any analysis involving distances.
% For instance, a structural variant that changes which gene a regulatory element affects. 

%%% Local Variables:
%%% mode: latex
%%% TeX-master: "main"
%%% End:


\section{Sequence Graphs}
A DNA-sequence can be represented by a sequence of letters from the alpahbet (A, C, G, T).
In nature it is common for DNA-sequences to be highly similar to each other, only separated by small variations caused by mutations.
The most common such mutations are single base-pair mutations, commonly called single-nucleotide-polymorphism (SNP), and insertions and deletions of short subsequences.
The most common format for representing similiar DNA-sequences are a block format where deletions are represented by a ‘blank’ symbol ‘\_’.
This format has a redundancy in the representation of the shared parts of the sequences. 
This works fine for small sets of short sequences, but for sequences on the genomic scale the redundancy gets significant, and the number of ‘-’ symbols needed gets two big. 
Sequence graphs are an alternative to the block format for representing multiple similar sequences.
In its basic form  a sequence graph is a collection of nodes, representing nucleotides, and a set of edges representing neighbouring pairs of nucleotides.
A single nucleotide sequence can then be represented as the nucleotides of the sequence connected by edges.
Similarily two sequences that are separated by a single SNP can be represented as such… Any set of sequences represented in block format can be uniquely represented by a sequence graph.
In addition to the graph, a representation of which paths each sequence takes through the graph is needed in order to contain the same information as the block format.
Even without thes specific paths, the graph representation of the sequences contain meaningful information.
They succinclty sum up the variation between the sequences.
Also any path through the sequence graph is a possible combination of the sequences present that can represent a sequence obtained by recombination events from the available sequences.
Also each path through the graph represents a possible common ancestor from the graph. 

On a larger scale the block format is unconvenient due to the redundancy in the sequences.
It is then more common to use a single sequence as a reference sequence and then represent the different sequences only by the places they are different from the  reference.
This is common for example when representing mulitple genomes from the same species.
The variant common format? (VCF) format is a common such format that represents the.
It contains a list of variants represented by a position in the reference sequence, a subsequence from the refernce from that positions, and the alternative sequence that replaces the reference sequence.
In addition it can contain extra columns for known haplotypes, specifying which variants are present in each haplotype. 

Such vcf representations can also be represented uniquely by a sequence graph, by first representing the reference sequence as  a linear sequence graph and then adding to the graph the nodes and edges to represent the variants.
It is also here necessary to represent the haplotypes as paths through the graph in order to keep the haplotype information.

The most common variations (SNPs and indels), leads to directed acyclic graphs (DAG) when converted to a sequence graphs.
However other types of variants does not have this property and thus leads to more complicated graphs. 

Other types of variants
Large scale variants that affects the ordering of the nucleotides are not well suited to being represented by DAGs.
An example of this is transpositions, where a subsequence of DNA is moved to another location in the genome.
This can be represented as a DAG by adding a new variant in the graph covering the whole sequence from covering both the old an new postions of  the subsequence.
However this can lead to much redundancy since the sequence between the two positions will be represented twice.
The other alternative is to only add new edges to the graph, representing the sequence when the substring has been moved.
This will however break the acyclisity of the graph, and thus complicate most operations one would do on the graph. 

Another case which will lead to redundancy if represented as a graph is reversals.
Here a piece of the DNA is reversed leading to the subsequence being substituded with its reverse complement.
Adding a subsequence in the graph representing the reverse compliment is inderictly redundant.
Even though the reverse compliment is not included as a path in the graph, it is directly deducible from a sequence in the graph.
In order to represent suche reversals, and other things needing the reverse compement, the concept of a side graph has been introduced.
In a side graph, all the nodes representing nucleotides have to sides and an edge is defined as going from one side of a node to a side of another node.
One node-side represents the nucleotide, while the other side represents the complement in the other reading direction. 

Applications
Multiple sequence alignment
An early application of sequence graphs was in sequence alignment.
The application made a sequence graph of the pairwise alignments and also made it possible to align a sequence graph to another sequence graph.
A central theme in this application was that if an an alignment of two sequences minimizes some edit distance between them, then any path through the sequence graph representing this alignment will represent a possible common ancestor of the two sequences that minimizes the combined edit distance to the two sequences.
Thus in an iterative pairwise joining scheme, one can in each step achieve a sequence graph that represents the most likely common ancestor and then align these sequence graphs to each other.
Here, the graph representation is clearly intuitive and benefitial.
The fact that each possible path through the graph represents something meaningful makes the graph format very succinct and benefitial for this application. 

Mapping
In recent years, mapping reads to such sequence graphs have gathered much attention.
The process of mapping reads to a reference sequence is trying to find out where in the reference sequence a read is from and usually involves an index that can quickly look up subsequences from the reference.
Common such indexes for linear reference sequences are the FM-index, that uses the burrows wheeler transform to look up subsequences in linear time, and kmer based indexes that can look up subsequences of constant length in constant time.
Both of these types of indexes encounters problems when applied to a graph, due to the combinatorial growth of possible subsequences.
The GCSA2 index is a kmer-based index able to index a  generic sequence graph, but needs to prune out variants in complex regions in order to restrain the combinatiorial growth in kmers.
It also requires significantly more computaional time and memory to create the index than for linear references.
Another problem with mapping to a graph based references is that combining several subsequence matches is also hard. 
The early attempts at graph-mapping had the goal of finding any path in the graph.
It is however doubful that this is the right approach due to two facts.
First when including variants from many individuals in the graph, a region can be filled up with variants from many different samples where none of them appear in the same individual.
Thus a region of the graph can include very many paths, where just a marginal fraction of them actually represents sequences that are seen in the samples or could be attained by a small set of recombinations.
This is problematic since sequences will have an increased chance of mapping to such areas.
The mismapping in itself is an issue, but this will also lead to a bias toward such regions, so any downstream analysis of the data might be severly compromised. 
A possible remedy for this is to only search in sequences represented by a real haplotype, that is using the haplotype path information in the mapping. This avoids the problem both of combinatorial growth of subsequences and also that of mapping sinks.
It is however a question of whether the graph format is meaningful in this context, as the mapping is in reality linear. 

Downstream analysis
The possiblity to map reads to a variation graph leads to possibilities for downstream analysis.
The most natural is to call variants.
The process of variant calling is using mapped reads to determine in which parts of the sequence of the sample differs from the sequence in the  reference.
In this case, graph mapped reads can be advantageous.
%%% Local Variables:
%%% mode: latex
%%% TeX-master: "main"
%%% End:

\section{Formalism}
We define a \emph{simple sequence graph} $SG = \set{G, s}$ over an alphabet $\alphabet$ is a graph $G=\set{V, E}$ and a label function $s: V \rightarrow \alphabet$ labeling each vertex with a letter from $\alphabet$. The set of all sequences in $SG$ is 
$$\Sigma(SG) = \set{s(v_1), s(v_2), , , s(v_n) s.t. v_i \in V \wedge (v_i, v_{i+1}) \in E}$$

Similarily a \emph{compacted sequence graph} $CG = \set{G, s}$ over alphabet $\alphabet$ is a graph $G$ and label function $s: V \rightarrow \Sigma^* / \epsilon$ labelling each vertex to a non-empty finite string over $\alphabet$. The set of all sequences in $SG$ is
\begin{align*}
  \Sigma(CG) &= \Sigma_0(CG) \cup \Sigma_1(CG)\\
  \Sigma_0(CG) &=\set{s(v)_{a:b} s.t. v \in V \wedge 0<a<b\leq\size{s(v)}}\\
  \Sigma_1(CG) &= \set{(s(v_1)_{a:}, s(v_2),,, s(v_n)_{:b}) s.t. v_i \in V \wedge (v_i, v_{i+1}) \in E \wedge a>0 \wedge b\leq \size{v_n}}
\end{align*}

For DNA-sequence graphs over the alphabet $\alphabet=\set{A, C, G, T}$ and the reverse compliment function $rc: \alphabet \rightarrow \alphabet$, we get all sequences reverse sequences by
$$\Sigma_{rc}(SG) = \set{(rc(s_n), rc(s_{n-1}), , , rc(s_1)) \forall s=(s_1, s_2,,,s_n) \in \Sigma(SG)}.$$
and the set of all possible sequences on either forward or reverse strand is given by $\Sigma_{all}(SG) = \Sigma(SG) \cup \Sigma_{rc}(SG)$. 

In order to represent reversals succinctly one needs a generalization of the sequence graph as given in~\cite{vg}.
A \emph{reversible sequence graph} is given by $RG = \set{V, E}$, where $E \subseteq {(V \times \set{-1, 1})^2}$.
Here the possible sequences in the graph is represented by:
\begin{align*}
  \Sigma(RG) &= \set{s(v_1, d_1), s(v_2, d_2), , , s(v_n, d_n) s.t. v_i \in V \wedge ((v_i, d_i), (v_{i+1}, d_{i+1})) \in E}\\
  s(v, d) &=\begin{cases} s(v_1), & \text{if} d = 1\\
                        rc(s(v_1), & \text{if} d = -1\\
\end{cases}
\end{align*}

%%% Local Variables:
%%% mode: latex
%%% TeX-master: "main"
%%% End:

The alignment methods described in the previous section do not scale well to large sequences.
Aligning a read $Q[:m]$ to a reference genome $R[:N]$ will require computing $mN$ values.
A typical NGS sequencing experiment typically yields hundreds of thousands of reads of length 100 which needs to be located in the human reference genome of length 3 billions.
This would need to calculate $10^{16}$ values, which is impossible? even on modern computers. 

This has led to much development the last 13 years in developing alignment methods that avoid the complexity of the exact dynamic programming methods.
This has been achieved mainly by two measures, creating searchable indexes of the reference genome and using heuristics to limit the search space of possible alignments.

In the following is a brief description of the developments in this field, with focus on the aligner used in this thesis' projects: BWA-mem~\cite{bwamem}.
After this we will look at the methods used in for graph alignment with focus on \emph{vg}.

\subsection{Linear Mapping}
As in section~\ref{linmap}\TODO{Define this explicitly in alignmetn}, we define the mapping problem as finding the interval $i_R$ for wich the edit distance $D(R[i_r], Q)$ is smallest, and the alignment $A(R[i_R], Q)$.
\subsubsection{Indexing}
An essential tool for quick alignment is to have a searchable index of the reference genome which is capable of returning sets of interval-pairs from the reference and query sequence for which the reference sequence is identical or near identical to the query sequence.
The indexes used by different tools vary, but can mainly be divided into fixed-length(kmer) indexes~\cite{minimap,,,} and variable-length indexes~\cite{bowtie, bwasw,,,}.
Fixed length indexes typically uses hash tables to store the location of all, or subsets of all, kmers \TODO{define kmer} in the reference sequence.
Variable length indexes usually uses a variation of the \emph{full text minute index} (FM-index)~\cite{fm}, described below.
The main idea covered here is the seed-and-extend paradigm. This idea consists of finding a set of exact matches to the reference sequence, and using these as anchors for using dynamic programmic based alignment of the query.
We let $EM(Q, R)$ be the set of all exact matches, represented by the tuples $\buildset{(q, r, l)}{Q[q:q+l]=R[r:r+1]}$.
Since the dynamic programming methods can still be computationally expensive, the goal is to make the seeds as small a subset of exact matches as possible, while still making sure that the true alignment covers one of the seeds.

\subsubsection{FM-index}
The FM-index uses a succinct representation of suffix array~\cite{suffixarray} and Burrows-Wheeler transform~\cite{BWT} in order to find exact matches of a query string $S$ in $O(\size{S})$ time (see figure~\ref{fig:FM}).
Thus if a read has one or more exact matches in the reference sequence it can be found directly using the FM-index.
For inexact matches, it can also be used to find an exact alignment of $Q$ and $R$ in $O(\size{R}^{0.628}\size{Q})$ time~\cite{bwtsw,bwalong}.
Some algorithms also use the FM-index to find exact matches for permutations of the query sequence~\cite{bowtie1, bwashort}, but for longer reads the search space gets too big when allowing for indels.
\begin{figure}
  \tikzpicture
  \matrix (ref) [matrix of math nodes, align=right, row sep=0.2em,column sep=0.2em, minimum width=1em, minimum height=1em, nodes={align=right}, ]{
C & T & A & \textcolor{green}{G} & \textcolor{blue}{C} & \textcolor{red}{T} & \textcolor{green}{G} & C & A & \textcolor{red}{T} & \textcolor{green}{G}\\ 
0 & 1 & 2 & 3 & 4 & 5 & 6 & 7 & 8 & 9 & 10\\};
\node[left=0.2 of ref-1-1] {R:};
\draw (ref-1-7)--(ref-1-6) --(ref-1-5);
\draw (ref-1-11)--(ref-1-10);
\matrix (fm) [matrix of math nodes, align=right, row sep=0.2em,column sep=0.2em, minimum width=1em, minimum height=1em, nodes={align=right}, below=of ref]{
% SA & F & & L & $A_i$ & $C_i$ & $G_i$ & $T_i$ \\
- & \$ & CTAGCTGCAT & $G_0$ & 0 & 0 & 0 & 0\\ 
2 & $A_0$ & GCTGCATG\$C & $T_0$ & 0 & 0 & 1 & 0\\ 
8 & $A_1$ & TG\$CTAGCTG & $C_0$ & 0 & 0 & 1 & 1\\ 
7 & $C_0$ & ATG\$CTAGCT & $G_1$ & 0 & 1 & 1 & 1\\ 
0 & $C_1$ & TAGCTGCATG & \$ & 0 & 1 & 2 & 1\\ 
\textbf{4} & $C_2$ & TGCATG\$CTA & $G_2$ & 0 & 1 & 2 & 1\\ 
10 & $G_0$ & \$CTAGCTGCA & $\textcolor{red}{T_1}$ & 0 & 1 & 3 & 1\\ 
6 & $G_1$ & CATG\$CTAGC & $\textcolor{red}{T_2}$ & 0 & 1 & 3 & 2\\ 
3 & $G_2$ & CTGCATG\$CT & $A_0$ & 0 & 1 & 3 & 3\\ 
1 & $T_0$ & AGCTGCATG\$ & $C_1$ & 1 & 1 & 3 & 3\\ 
9 & $T_1$ & G\$CTAGCTGC & $A_1$ & 1 & 2 & 3 & 3\\ 
5 & $T_2$ & GCATG\$CTAG & $\textcolor{blue}{C_2}$ & 2 & 2 & 3 & 3\\};
\node[above=0.2 of fm-1-1] {SA};
\node[above=0.2 of fm-1-2] {F};
\node[above=0.2 of fm-1-4] {L};
\node[above=0.2 of fm-1-7] {OCC};

\matrix (query) [matrix of math nodes, align=right, row sep=0.1em,column sep=0.2em, minimum width=1em, minimum height=0.5em, nodes={align=right}, below=of fm]{
\textcolor{blue}{C} & \textcolor{red}{T} & \textcolor{green}{G}\\};
\node[color=green, draw, dashed, fit=(fm-7-2) (fm-9-2)] {};
\node[color=red, draw, dashed, fit=(fm-11-2) (fm-12-2)] {};
\node[color=blue, draw, dashed, fit=(fm-6-2)] {};
\draw[dashed, color=red] (fm-7-4) -> (fm-11-2);
\draw[dashed, color=red] (fm-8-4) -> (fm-12-2);
\draw[dashed, color=blue] (fm-12-4) -> (fm-6-2);
\node[left=0.2 of query-1-1] {Q:};
% \node[left=of ref] {R:};
% \node[left=of query] {Q:};


%%% Local Variables:
%%% mode: plain-tex
%%% TeX-master: t
%%% End:

  \endtikzpicture
  \label{fig:FM}
  \caption{Illustration of backward extension using the last-first (LF) property of the FM index.
    The rows represent sorted suffixes of the reference. The SA column holds the indices in the reference sequence for each suffix, the $F$ column holds the first character of each suffix, while the $L$ column holds the preceding character of each suffix. Subfixes gives the occurances of each character in each column. I.e. $T_i$ is the $i$th occurance of $T$ in that column. Finding the string \emph{CTG} is done by (1) starting with the range of all $G$'s in the F column; (2) finding all $T$'s in this range in the $L$ column; (3) mapping by occurance number those $T$'s to the F column; (4) mapping the $C$'c in the current range in the L folder by occurance number. The end result is in this case a single row which represents position $4$ in the reference sequences. 
}
\end{figure}

In order to handle indels in longer reads, the main methodology is to use the FM-index to find exact matches between substrings of the query and reference sequence, called seeds.
And then aligning the reads using dynmaic programing to intervals surrounding the seed matches on the reference sequence~\cite{bowtie2}, often called the seed-and-extend paradigm.

One way of finding seeds is to find Maximal Exact Matches (MEM)~\cite{longmem, origmem}.
Maximal Exact Matches are Exact Matches that cannot be extended in either direction, ie. 
\[
  MEM(Q, R) = \buildset{(q, r, l) \in EM}{(q, r, l+1) \notin EM \wedge (q-1, r-1, l+1) \notin EM}
\]
BWA-mem furthers this concept to SuperMaximal Exact Matches (SMEM)~\cite{origsmem}.
A SMEM is a MEM where the query interval cannot be extended either side and still yield an Exact Match~\ref{fig:smem}.
\[
  SMEM(Q, R) = \buildset{(q, r, l) \in EM}{\nexists r^*[(q, r^*, l+1) \notin EM \vee (q-1, r^*-1, l+1) \notin EM}
\]
\begin{figure}
  \tikzpicture
  
\matrix (seq) [matrix of math nodes, align=right, row sep=0.5em,column sep=0.5em, minimum width=1em, minimum height=1em, nodes={align=right}, ]{C & T & A & G & C & T & G & C & A & T & G\\ 
G & A & T & C & G & A & C & G & T & A & C\\};
\node[left=0.2 of seq-1-1] {$R\rightarrow$};
\node[right=0.2 of seq-2-11] {$\leftarrow\bar{R}$};
\matrix (q) [matrix of math nodes, align=right, row sep=0.5em,column sep=0.5em, minimum width=1em, minimum height=1em, nodes={align=right}, below=of seq]{T & A & C & T & G & A & T & G\\};
\node[color=red, draw, dashed, rounded corners, fit=(q-1-3) (q-1-5)] {};
\node[color=red, draw, dashed, rounded corners, fit=(seq-1-5) (seq-1-7)] {};
\node[color=green, draw, dashed, rounded corners, fit=(q-1-1) (q-1-2)] {};
\node[color=green, draw, dashed, rounded corners, fit=(seq-1-2) (seq-1-3)] {};
\node[color=green, draw, dashed, rounded corners, fit=(seq-2-2) (seq-2-3)] {};
\node[color=blue, draw, dashed, rounded corners, fit=(q-1-6) (q-1-8)] {};
\node[color=blue, draw, dashed, rounded corners, fit=(seq-1-9) (seq-1-11)] {};
\node[color=blue, draw, dashed, rounded corners, fit=(seq-2-8) (seq-2-10)] {};
  \endtikzpicture
  \label{fig:smem}
  \caption{SMEMs found between query $Q$ and both strands of reference $R$. Note that the MEM $(q, r, l)=(2, 0, 2)$ (CT) is not a SMEM, since it is contained in the SMEM $(q, r, l) = (2, 4, 3)$}
\end{figure}

SMEMS are natural to use as seeds as they cover for each subsequence in $Q$ the longest exact match in $R$.
They are however vulnerable for spurious long matches hiding shorter exact matches.
To account for this BWA-mem allows an option to split long SMEMs into shorter MEMs if they are longer than a certian threshold.
Splitting SMEMs like this increases accuracy, since it increases the number of seeds, but can negatively affect performance. 
In order to find SMEMS, an adaption of the FM index is used, the FMD index, where $FMD(R) = FM(R \concat \bar{R}$.

\subsubsection{Prioritizing and Extending}
The seeds found from the index are next used as seeds for DP based alignment.
$Q$ is then aligned against an area around $(r, r+l)$ for each seed $(q, r, l)$.
Since this is a computationally expensive step, further limiting the set of seeds is advanteous.
BWA-mem does this by \emph{chaining} the seeds.
This is done by grouping approximately colinear, nearby seeds into chains, and removing small chains that overlap with larger chains.
Approximately colinear means that $\abs{(q_1-q_2) -(r_1-r_2)}<w$ for some set threshold.

\subsection{Graph Mapping}
Mapping to a graph based reference is similar to the linear case, except that instead of finding a linear interval, the goal is to find a graph interval $i_r$ such that $D(\slabel(G_R(i_r), Q))$ is minimized. 

% estimating a liner interval $(\hat{s}, \hat{e})$, a graph interval $(\hat{s}, \hat{e}, \hat{v})$ is estimated.
It is however deceptively more complicated. 
Firstly because the number of subsequences in a sequence graph grows exponentially with the complexity of the graph. 
And secondly since chaining subsequence-matches is more complicated due to the possible existance of multiple paths between two matches.
\emph{vg}~\cite{vg} has been at the forefront of mapping to a graph reference, showing that it can lead to better mapping accuracy than BWA-mem. 
% But it is still too slow and memory- and disk-consuming for widespread use.
Below is a brief description of the methodology used by \emph{vg}.

\subsubsection{\emph{vg}}
\emph{vg} uses much the same methodology as BWA-mem to align reads.
It uses the GCSA2-index to find SMEMs, uses chain filtering to filter the SMEMs, and uses the graph adaption of Smith-Waterman to extend the seeds.
\emph{vg} is able to align reads to more complicated graph structures than the simple directed sequence graphs considered in this thesis.
For simplicity the descriptions below will be contained to simple graphs, which entails that the GCSA index described is GCSA1\cite{gcsa1} which is only able to index directed sequence graphs.

\subsubsection{GCSA}
The GCSA-index~\cite{gcsa, gcsa2} is a generalization of the FM-index, where arbitrary-length sequences can be looked up in a sequence graph.
Originally constructed to work on acycilic sequence graphs, gcsa2 extends the functionallity to general variation graphs.
For simplicity we will here constrain the discussion to acyclic graphs.

GCSA uses the same concept of $LF$ mapping as the FM-index does. Here the F column holds the label of each node in the graph, sorted by the suffix starting from that node. The L column holds the label of the corresponding node's predecessor nodes. The problem with this setup is that there are many suffixes starting from each node, depending on which path is taken in the graph. To resolve this problem, the graph needs to be expanded so that for each node, all suffixes starting from that node shares a prefix that is not found from any other node in the graph(see figure~\ref{fig:gcsa}).

For complicated regions in the graph, this expansion procedure gets too costly, and the GCSA index therefore needs to prune edges in such areas in order to be able to index it. This means that not all possible sequences in the graph gets indexed. Even after pruning, this procedure is costly and makes the GCSA index significantly slower to create, and more memory consuming than the FM-index.

\begin{figure}
  \tikzpicture
  \matrix (seq) [matrix of math nodes, align=right, row sep=0.5em,column sep=0.5em, minimum width=1em, minimum height=1em, nodes={align=right}, ]{$G^0$ & $C^1$ & $A^2$ & $C^4$ & $G^5$\\ 
 &  & $T^3$ &  & \\};
\draw (seq-1-1) -- (seq-1-2);
\draw (seq-1-1) -- (seq-1-2);
\draw (seq-1-2) -- (seq-1-3);
\draw (seq-1-2) -- (seq-2-3);
\draw (seq-1-3) -- (seq-1-4);
\draw (seq-2-3) -- (seq-1-4);
\draw (seq-1-4) -- (seq-1-5);
\draw (seq-1-4) -- (seq-1-5);

\matrix (seqe) [matrix of math nodes, align=right, row sep=0.5em,column sep=0.5em, minimum width=1em, minimum height=1em, nodes={align=right},right=of seq]{
$G^0$ & $C^1$  & $A^2$ & $C^5$ & $G^6$\\ 
  & $C^3$  & $T^4$ &   &  \\};
\draw (seqe-1-1) -- (seqe-1-2);
\draw (seqe-1-1) -- (seqe-2-2);
\draw (seqe-1-2) -- (seqe-1-3);
\draw (seqe-2-2) -- (seqe-2-3);
\draw (seqe-1-3) -- (seqe-1-4);
\draw (seqe-2-3) -- (seqe-1-4);
\draw (seqe-1-4) -- (seqe-1-5);
\draw (seqe-1-4) -- (seqe-1-5);
\matrix (gcsa) [matrix of math nodes, align=right, row sep=0.5em,column sep=0.5em, minimum width=1em, minimum height=1em, nodes={align=right},below=of seq]{
SA & F & L\\
  & $\$$ & $G_0^5$ \\
2 & $A_0^2$ & $C_0^4$ \\
1 & $C_{0,1}^1$ & $G_1^0$ \\
4 & $C_{2}^4$ & $A_0^2, \textcolor{red}{T_0^3}$ \\
5 & $G_{0}^5$ & $C_1^1$ \\
0 & $G_{1}^0$ & $\$$ \\
3 & $T_0^3$ & $\textcolor{blue}{C_2^1}$\\
};
\draw[dotted] (gcsa-5-2) -- (gcsa-5-3);
\draw[dotted, color=red] (gcsa-5-3)  -- (gcsa-8-2);
\draw[dotted] (gcsa-8-2) -- (gcsa-8-3);
\draw[dotted, color=blue] (gcsa-8-3) -- (gcsa-5-2);

\matrix (gcsag) [matrix of math nodes, align=right, row sep=0.5em,column sep=0.5em, minimum width=1em, minimum height=1em, nodes={align=right},below=of seqe]{
SA & F & L\\
  & $\$$ & $G_0^6$ \\
2 & $A_0^2$ & $C_0^1$ \\
1 & $C_0^1$ & $G_1^0$ \\
5 & $C_1^5$ & $A_0^2, \textcolor{red}{T_0^4}$ \\
3  & $C_2^3$ & $\textcolor{yellow}{G_2^0}$ \\
7 & $G_{0}^6$ & $C_1^5$ \\
\textbf{0} & $G_{1,2}^0$ & $\$$ \\
4 & $T_0^4$ & $\textcolor{blue}{C_2^3}$\\
};
\draw[dotted] (gcsag-5-2) -- (gcsag-5-3);
\draw[dotted, color=red] (gcsag-5-3)  -- (gcsag-9-2);
\draw[dotted] (gcsag-9-2) -- (gcsag-9-3);
\draw[dotted, color=blue] (gcsag-9-3) -- (gcsag-6-2);
\draw[dotted] (gcsag-6-2) -- (gcsag-6-3);
\draw[dotted, color=yellow] (gcsag-6-3) -- (gcsag-8-2);
%\draw[dotted] (gcsa-5-2) -- (gcsa-5-3) -- (gcsa-8-2) -- (gcsa-8-3);\\


\node[above=0.2 of seq] {Invalid};
\node[above=0.2 of seqe] {Valid};
% \node[above=of seq] {Invalid};

\node[color=green, draw, dashed, fit=(gcsa-4-2) (gcsa-5-2)] {};
\node[color=red, draw, dashed, fit=(gcsa-8-2)] {};
\node[color=blue, draw, dashed, fit=(gcsa-5-2)] {};

\node[color=green, draw, dashed, fit=(gcsag-4-2) (gcsag-6-2)] {};
\node[color=red, draw, dashed, fit=(gcsag-9-2)] {};
\node[color=blue, draw, dashed, fit=(gcsag-6-2)] {};
\node[color=yellow, draw, dashed, fit=(gcsag-8-2)] {};


\matrix (query) [matrix of math nodes, align=right, row sep=0.1em,column sep=0.2em, minimum width=1em, minimum height=0.5em, nodes={align=right}, below=of gcsa]{
\textcolor{yellow}{G} & \textcolor{blue}{C} & \textcolor{red}{T} & \textcolor{green}{C}\\};
\node[left=0.2 of seq] {$G_R$:};
\node[left=0.2 of gcsa] {GCSA:};
\node[left=0.2 of query] {$Q$:};


%%% Local Variables:
%%% mode: plain-tex
%%% TeX-master: t
%%% End:

  \endtikzpicture
  \label{fig:gcsa}
  \caption{
    Valid and invalid GCSA index.
    In the first graph, the two edges from $C$ to $A$ and$T$ makes a direct GCSA index impossible due the other $C$ with an edge to a $G$.
    Both $T$ and $G$ needs to point to the same row in the index.
    In the valid example, the $C$ node have been duplicated.
    The bifurcation now happens at an earlier point, and the $GC$ now present is unique so that the the LF mapping is valid.}
\end{figure}

Since the gcsa index provides the same functionality as an FM-index, it can be used to find SMEMs in much the same manner. These SMEMs are in \emph{vg} used as seeds. 

GCSA2 solves the problem in a slightly different way that allows for indexing general variation graphs.
This is based on succinct de-Bruijn graph~\cite{debruijn, succinctdebruijn} structure which sets a limit to the length of the unique prefixes by allowing for false positive edges to occur in the index.
This means that results from an index lookup needs to be validated by traversing the graph. 

\subsubsection{Filtering and Extending}
\emph{vg} employs a similar method as BWA-mem for filtering out seeds: chaining the seeds and filtering out small chains that overlap bigger chains.
The chaining procedure is however more complicated on a graph, and involves clustering the seeds by position in addition to a Markov Model based method to find approximately colinear seeds. 

% At the heart of \emph{vg} is the concept of Variation Graphs. A Variation Graph is a side graph togheter with a set of paths that represents the sample sequences. Since it uses a side graph and not a simple sequence graph it can represent SNPs and indels in addition to the large scale variants discussed above. This however comes at the price of complexity. A key functionality of \emph{vg} is to map reads to the the side graph. This is done similarily to how BWA-mem maps read to a linear reference sequence, by finding matches for subseqeunces and chaining these togheter. The process is more involved due the the complexity of graphs. 
% \TODO{vg chaining}


%%%%%%%%%%%%%%%%%%%%%%
% Mapping a read $r \in \alphabet^*$ to a linear reference genome $G \in \alphabet^*$ can be seen as solving the problem of finding an interval $M(r | G) = (\hat{s}, \hat{e})$ that minimizes some edit distance $D(r, G_{\hat{s}:\hat{e}}$ as well as returning an uncertainty estimate of the mapping $\epsilon(r, G, (s, e))$.
% Given no restraints on computation time or memory usage, a local alignment of $r$ versus $G$ finds the exact minimum \TODO{which distance}.
% However since this algorithm has complexity $O(\size{r}\size{G})$, this is not possible in practice when dealing with million of reads and reference genomes with a length in the billions.
% Thus indexing and further approximations are needed to meet the demands of modern bioinformatics.
% A common approach is to generate an index that allows for efficient searching of subsequences, and to combine the matches from subsequences into a match for the whole sequence $r$.
% The projects in this thesis have used $BWA-mem$ as the program to map reads to a linear reference genome. 
% In the following a brief description of the methods used in $BWA-mem$~\cite{bwa} is explained.

% While the creation of the index only needs to be done once for a reference genome, using it to find subsequence matches needs to be perfomed hundreds of thousand times.

% 
% 
% The BWA-mem uses the strategy of first finding exact matches to subsequences followed by chaining these subsequences together. Finding the exact matches is done using an FM-index.
% 
% An FM index~\cite{fmindex} is a data structure that allows lookup of a sequence $s$ in a reference sequence $R$ in $O(\size{s})$ time.
% It is based on a lexiographic sort of the suffixes of $R$ \TODO{BWT?} and the index structure contains representations of two character vectors.
% One for the first letter of each suffix in sorted order and one for the letter preceding each suffix. The power of the FM-index comes from properties of these columns.
% \begin{itemize}
% \item The letters in $L$ are sorted, so that it can be represented by the offset of the firt occurance of each letter in the alphabet
% \item Each letter $r_i$ in $R$ represents a new suffix, $r_iS_i$.
% \item If $r_i$ is the $n$th occurence of that letter in $R$, the suffix $r_iS_i$ is represented by the $n$th occurance of $r_i$ in $L$.
% \end{itemize}
% These properties makes it possible to start from the last letter of a query string and traverse the FM-index in order to find  all occurances of that string in the reference, or finding the longest suffix of the string that is precent in the reference string. 
% 
% \subsubsection{BWA-mem}
% The FM-index can be used to find \emph{Super-Maximal Exact Matches} (SMEMs). These are exact matches of subsequences of the query to the reference, with the property that they cannot be extended further and are not contained in another. The set of SMEMs on 
% \TODO{mem-chaining}
% 

% , but first a brief discussion of variation on genome scale. 

% 
% pect to a singe reference sequence. Thus instead of representing each contributing sample as a row in a block, as in multiple sequence alignment, one represents each variant with a position in the reference sequence, a subsequence of the reference from that position, and the alternative sequence that replaces that sequence. The full sequence of a sample can then be represented as a sample of variant ids. The most common format for representing such variants and sample sequences is the Variant Call Format (VCF)\cite{vcf}. 
% 
% 
% 
% quely by a sequence graph. This is done by first representing the reference sequence as  a linear sequence graph and then adding to the graph the nodes and edges to represent the variants. The sample sequences can also here  be represented as paths through the graph. 
% 
% erted to a sequence graphs.
%  thus leads to more complicated graphs. 
% 
%  represented by DAGs.
% r copied (\TODO{only copied?}) to another location in the genome.
% nce from covering both the old an new postions of the subsequence.
% represented twice.
% e substring has been moved. This will however break the acyclisity of the graph, and thus complicate most operations one would do on the graph. 
% 
% 
% rse complement.
% 
% cible from a sequence in the graph.
% cept of a side graph has been introduced.
% s going from one side of a node to a side of another node.
%  other reading direction.


%%% Local Variables:
%%% mode: latex
%%% TeX-master: "main"
%%% End:

Peak Calling is an important step in the CHiP-seq experiment pipeline.
The sequencing in a ChIP-seq experiments yields reads that tend to come from DNA-fragments with the transcription factor bound to it. After mapping these reads to the reference genome, the result is a set of genomic intervals which should be in the vicinity of the binding site. Peak calling is the process of using these intervals to predict the actual binding sites. 

Calling peaks faces two main problems, noise and bias. Noise can come from either of the previous steps of the experiment: isolating DNA fragments with the TF bound to it, sequencing the fragments, or mapping the resulting reads to the reference genome. The bias can be either biological, in that certain regions of the genome are over/under-represented in the sample independently of the TF, or from the mapping, in that certain regions of the reference genome are more or less likely to be mapped to. In addition to this, the sequenced reads do not always contain the actual binding site, but rather comes from the beginning of a fragment containing the binding site. 

In light of this, peak callers seek to recognize the pattern expected from a true binding site: that the mapped intervals should generally be close to and point towards the true binding site.
Several algorithms exists for this~\cite{SPP, MACS2, MACS, }, of which MACS2 is one of the most widely used. As it forms the basis for Graph Peak Caller (paper 2), the algorithm is described in some detail in figure~\ref{fig:macs}:

\begin{figure}
  \includegraphics{figures/macs2}
  \caption{Illustration of the main MACS2 algorithm. After reads are mapped to the reference genome, all mapped intervals are extended to match the estimated fragment length. A pileup is then created based on the coverage of each position, and positions with a high enough coverage are marked as reads}
  \label{fig:macs}
\end{figure}

% 
% 
% 
% 
% 
% 
% 
% 
% 
% 
% 
% 
% 
% 
% 
% 
% 
% 
% 
% 
% 
% 
%  extension size.
% 
% oo many ps}. Too adjust for multiple testing, a final set of q values is calculated as $q_i = p_iN_i$ where $N_i = \size{\buildset{k\in\set{1, 2,,,\size{G}}}{p_k\leq p_i}}$. The q values are thresholded on a given significance level $\alpha$ so we get $T(i) = q_i\leq \alpha$. 
% 
% 
% 
% 
% 
% 
% 
% 
% 
% 
% 
% 
% 
% 
% 
% 
% 
% 
% 
% 
% 
% 
% 
% 
% 
% 
% h $f$. Both steps can be made using the same function.
% 
% 
% 
% 
% 
% 
% 
% 
% 
% 
% 
% 
% 
% 
%  of positions in the graph.
% rd$ by \code{node_offsets[v]+o}. The inverse conversion of a coordinate $i \in \linearcoord$ is done by \code{v=searchsorted(node_offset, i); o = i-node_offsets[v])}
% 
% 
% 
% 
% 
% 
% 
% 
% 
% 
% 
% 
% 
% 
% 
% 
% 
% 
% 
% 
% 
% 
% 
% 
% 
% 
% 
% 
% 
% 
% 
% 
% 
% 
% 
% 
% 
% 
% 
% 
% 
% 
% 
% 
% 
% 
% 
% onger than the estimated fragment length.
%%% Local Variables:
%%% mode: latex
%%% TeX-master: "main"
%%% End:

  \section{The Importance of Non-Redundancy}
  The simplicity of working with acyclic sequence graphs compared to generalized sequcence graphs with possible cycles are big, both in terms of algorithmic effiency and inituitvity.
  Beacuse of this a compelling argument needs to be made for dealing with generalized sequence graphs.
  Here we will disucss the benefit of these more complex structuers, both in terms of mapping and interpreting intervals and locations on the graphs. 
\subsection{Mapping}
  A key element of the read mapping procedure is to filter out reads that map with similar scores to multiple places in the reference.
  Repeats in the reference of the same subsequences is thus problematic, since often a read mapping to the subsequence, will have mathces in many of the repeats.
  The result of this is that the read will get a low mapQ score.
  This will however lead to a loss of information, since a read mapping to a reapeated subsequence can still give information about where in the genome it belongs, and it will also be possible to see if there is a variation in the read from the references subsequence. 
  In this case, representing the repeated subsequences as a cycle in the graph is benefitial, since as much information as possible about the read mapping is kept.
  A similar case can be made for representing reversals in generalized sequence graphs.
  In a simple graph, the reversed sequence would need duplicated representation, yielding two matches for each read.
  These multiple mapping issues could also be alieviated by including information in the mapping algorithm that specifies which regions are reversals and duplications of each other, akin to the alt-allele handling in BWA-mem.
  However representing the location of the resulting alignments would be problematic, which leads to the next benefit of genralized graph structures: representation.
  \subsection{Representing locations and intervals}
  The output from the read mapping is generally the location on the reference of the match, and a description of how the read sequence diverges from the reference sequences at that location.
  The location of the mappings can be used in downstream analysis.
  It is then beneficial if the location of the read gives as much information about the whereabout of the mapping as possible.
  Without the possibility of cycles, reapeats and reversals would need to be represented as either the set of locations representing the sequences, or a random location from this set.
\section{Data Structures and Algorithms}
The focus of this thesis is on the representation and handling of intervals and positions in compressed sequence graphs.
The nonlinearity of sequence graphs means that the complexity of interval representation and handling becomes an issue.
In a linear genome both representation and handling of intervals can usually be done in constant complexity.
An interval can be represented as simply a start and end position along with a specifier of direction $(s, e, d)$.
Similarily distances between two in intervals can be computed in constant time by
$$D(I_1, I_2) = \max(0, s_1-e_2, s_2-e_2)$$
and the overlap between two intervals by
$$\size{I_1 \cap I_2} = \max(\min(e_1, e_2) - \max(s_1, s_2), 0)$$.
Paper 1 discussed the implication of breaking this inherent simplicty when having a graphical and not linear reference.

In the general case a position on the graph needs to be represented by a tuple specifying the id of the node and the offset, and an interval needs a start position, 
an end position and the specification of each node covered by the interval.
This means that the memory cost of representing an interval grows linearly with the interval length.
Also, the distance between two positions is not uniquely defined, but depends on which path through the graph is taken between the two points.
And finding specific distances, such as the shortest or longest possible distance, does not in the general case have constant complexity, but rather depends on both the distance itself, and the complexity of the graph.

Also off particular importance to Graph Peak Caller was the operation of finding all positions within a distance $d$ from a position $P$. On a linear coordinate system, this is imply represented by the range $\left[P, P+d\right)$. However on a graph this constitutes finding each position reachable by a path of length $<d$ starting on $P$. Using a breadth first search, this can be on average obtained in $O(kd)$ time.

Graph Peak Caller suffered performance wise from these additional complexities. Where MACS2 can call peaks on \TODO{n} reads in the time frame of a couple of minutes on a standard laptop, GPC used close to a day on several CPUs and used more memory than available on a standard laptop. 

GPC2 introduces some simplifications to the graph structure that allowed it to run much closer to MACS2 running times.
Going away from \emph{vg}s data structure of nodes of maximum length of often $32$ and storing each variant as a separate node. 
It is worth to note that a single SNP in that format increases the number of nodes by 3 the number of edges by 4.
Coupled with the fact that the vast majority of variants are SNPs, this means that the majority of edges in the graph represents variants that does not affect the distance.
It was therefore possible to reduce the number of edges, and thereby the complexity of the graph $k$, by representing SNPs in a separate structure.

Throughout the project two open source python libraries were developed and optimized, in order to work efficiently with sequence graphs. \emph{Offsetbased Graph} provides basic functionality while \emph{Graph Peak Caller} provides methods necessary for doing peak calling on sequence graphs. The final implementations and algorithms is described below. 

\subsection{Sequence Graph}
The main data structure provided is a full graph, having as members a \emph{Graph}, describing the topology of the sequence graph, a \emph{Sequences} object, giving access to the sequence on each node and a \emph{Path} object, describing the path the linear reference takes through the graph. 

A \emph{Graph} object represents a graph $G = (V, E)$ and a node labeling $L: V \rightarrow \natural$ where the vertex set is assumed to be a dense set of integers starting at $1$, $V=\set{1, 2, , , N}$, and the labeling $L$ describes the sequence length of each node. The Graph object also contains a representation of the positions and sequence of all SNPs in a \code{SNP} object. 
  The vertex set $V$ and vertex labeling $L$ is represented as an array of length $N$ where each element is the sequence length of the corresponding vertex.
  The set of edges $E$ is described by a ragged array \code{adj_list} equivalent of an adjacency list so that \code{adj_list[i]}=$\set{v : v \in V \wedge (i, v) \in E}$. For convenience the graph object also contains a reverse adjacency list such that: \code{adj_list[i]}=$\set{v : v \in V \wedge (v, i) \in E}$. The implementation of the ragged arrays are described later in this section.
  The \code{SNP} class is a ragged array having as elements the location of all SNPs on all vertices. Such that $SNP[i] = \set{(p, i):} $ \TODO{describe SNP set}. 
The \code{Sequences} class is a ragged array where each element is the sequence of that node. \code{Sequences[i]=s(i)}. 
The \code{Path} class contains an array \code{node_ids} of length $n_p$ which describes which nodes the path traverses, and an array \code{distance_to_node} of length $n_p+1$ giving the distance along the reference path to each node in the path, such that \code{(distance_to_node[i], distance_to_node[i+1])} gives the interval vertex $i$ represents along the reference path.
\subsection{Creating from Variants}
We start with a reference sequence $G \in \alphabet^*$ and a set of variants described in the following.

We let $S \subset \integers _{<\size{G}} \times \alphabet $ be a set of SNPs  where $(i, c)$ represents a SNP at position $i$ to letter $c$. 
A set of insertions $I \subset \integers _{<\size{G}} \times \alphabet^*$ where $(i, s) \in I$ represents an insertion at position $i$ with sequence $s$.
A set of deletions $D \subset \integers _{<\size{G}} \times \natural^+$ where $(i, l) \in D$ represents an deletion at position $i$ with length $l$.

From a variation set $V=(S, I, D)$ we create the graph by the following method.
Breakpoints $B = \buildset{i}{\est{s}{(i, s) \in I}} \cup \buildset{i}{\est{l}{(i, l) \in D \vee (i-l, l) \in D}}$.
Let $(b_1, b_2, ..., b_{n_r+2})$ be the ordered elements of $B \cup \set{0, \size{G}}$. We then get a reference graph as
\begin{align*}
  G_r &= (V_r, E_r)\\
  V_r &= \set{1, 2,,, n_r}\\
  E_r &= \set{(i, i+1) | i \in \set{1,2,,,n_r-1}}\\
  s(i) &= G[b_i:b_{i+1}]
\end{align*}
The set of deletion edges is $E_d=\set{(i, j) | (b_i, (b_j-b_i)) \in D}$. 
Letting $I_s = (i_1, i_2,,, i_{\size{I}})$ be the sorted elements of $I$, the set of insertion vertices and edges are given by:
  \begin{align*}
    V_i &= \set{n_r+1, n_r+2,,, n_r+\size{I}}\\
    E_{-1} &= \buildset{(r-1, k)}{k \in \set{1, 2,,,\size{I}} \wedge b_r=i_k}\\
    E_{1} &= \buildset{(k, r)}{k \in \set{1, 2,,,\size{I}} \wedge b_r=i_k+}\\
    E_i &= E_{-1} \cup E_1\\
  \end{align*}
\TODO{Properly describe insertions.}
The full graph can then be made by $G = (V_r \cup V_i, E_r \cup E_i \cup E_d)$.
SNPs is then represented such that $SNP_i = \buildset{(p-b_i, c)}{(p, c) \in S \wedge b_i\leq p < b_{i+1}}$. \TODO{vertex labels}


\chapter{Summary of Papers}
In paper 1, we discussed the implications of using a graphical reference structure for representing and comparing genomic locations and intervals.
The work focused on how to obtain succinct and interpretable representations that were robust against changes to the reference graph topology. 
Paper 2 developed a new method of calling ChIP-seq peaks using a sequence graph as reference.
The method is an adaptation of MACS2 that uses the benefits of read-mapping to a variation graph to obtain more accurate peak calling of potential transctription factor binding sites. 
Paper 3 further developed the method of paper 2 to perform better on graphs constructed solely from SNPs and indels from a VCF-file. Introducing a new succinct data structure for variation graphs and obtaining similar speeds to MACS2.
Paper 4 looked into the potential for using a two step approach for read mapping using the benefits of mapping to a graph for estimating a genotype for the sample, before using a linear mapper to obtain a final mapping that is not suffer from the too big search space of graph mapping. 


%%% Local Variables:
%%% mode: latex
%%% TeX-master: main
%%% End:
