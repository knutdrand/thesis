The following sections introduces some notation needed for the next chapters. This notation will mainly follow a simplification of the notation used in~\cite{gcsa1}.
\subsection{Sequences}
Sequences are important in this thesis as they are the means to represent DNA-sequences.
We will here formally work with an alphabet $\alphabet_{DNA} = \set{A, C, G, T}$ and say that a \emph{sequence} of length $n$ over that alphabet is a tuple in the set $\alphabet_{DNA}^n$.
The set of all sequences of any length over $\alphabet_{DNA} = \set{A, C, G, T}$ is  represented by $\alphabet_{DNA}^*$.
The length of a sequence $S \in \alphabet_{DNA}^*$ is notated $\size{S}$.
The $i$th symbol in a sequence $s$ is notated $S[i]$ and a substring of $S$ starting at the $i$th symbol (inclusive) and ending at the $j$th symbol (exclusive) is denoted $S[i:j]$.
A \emph{prefix} of $S$, i.e. a substring of $S$ starting from the first symbol, of length $k$ is denoted $S[:k]$, while a \emph{suffix} of $S$ starting at the $k$th symbol is denoted $S[k:]$.
A \emph{kmer} of a sequence $S$ is any substring of $S$ of length $k$.

An \emph{interval} on a sequenence is a tuple of start and end position $(s, e)$.
For an interval $i_S=(s, e)$, the notation $S[i_S]$ will mean the subsequence between the start and end coordinate $S[s:e]$.

For convenience, a string of length $n$ is sometimes represented as $S[:n]$ to convey the size of the string.
Concatenation of two strings $S, T$ are represented as $S \concat T$, while the concatenation of a symbol $a$ to a string $S$ is denoted $Sa$, and a string to a symbol as $aS$.

For some algorithms a $\#$ symbol is added to the start and/or a $\$$ symbol is added to the end of the string.

\subsection{Graphs}
A graph is a tuple $G=(V, E)$ of vertices and edges, where the edges are pairs of vertices $E \subset V^2$.
We will here deal with directed graphs where we say that edge $(v_1, v_2)$ is an edge from $v_1$ to $v_2$.
A \emph{path} is an alternating sequence of vertices and edges $(v_1, e_1, v_2, e_2,,,e_{n-1}, v_n)$ where $e_i = (v_i, v_{i+1})$.
A \emph{cycle} is a path $(v_1, e_1, v_2, e_2,,,e_{n-1}, v_n)$ where $v_1=v_n$ and $n>1$, i.e. a path that starts and ends at the same vertex.
A \emph{directed acyclic graph} (DAG) is a directed graph in which there are no cycles.
We call the set of all paths starting at $v_s$ and ending at $v_e$ $\paths(v_s, v_e)$.
The \emph{predecessors} of a node $v$ are all nodes with an edge going to $v$, i.e: $\preds(v) = \buildset{w \in V}{(w, v) \in E}$.

\subsection{Sequence Graphs}
We define a \emph{sequence graph} $SG = (V, E, \slabel)$ over an alphabet $\alphabet$ as a graph $(V, E)$ and a label function $\slabel: V \rightarrow \alphabet$ labeling each vertex with a letter from $\alphabet$.
We refer to the label of a path $\slabel((v_1, e_1, v_2,,,e_{n-1}v_n))$ as the concatenation of the labels of its vertices $\slabel(v_1)\concat\slabel(v_2) ... \slabel(v_n)$.
The sequence graphs we look at here will have clearly defined start and end nodes, which will be denoted $v_s$ and $v_e$. If this were not the case, dummy nodes ($\#, \$$) can be inserted to the graph to make the start and end unambiguous.
The \emph{language} recognized by a sequence graph $\lang((G, \slabel))$ is the label of each path going from the start node to the end node, i.e:
\[
  \lang((G, \slabel)) = \buildset{ \slabel(p)}{p \in \paths(v_s, v_e)}
\]
A \emph{graph interval} is a path in the graph, and for a graph interval $i_G$, the notation  $SG[i_G]$ will mean the label of that path, i.e. $\slabel(i_G)$.

%%%Local Variables:
%%% mode: latex
%%% TeX-master: "main"
%%% End:
