\begin{figure}
  \begin{tikzpicture}[decoration={coil},
dna/.style={decorate, thick, decoration={aspect=0, segment length=1.0cm}},
protein/.style={ellipse, draw=white, minimum width=1cm, minimum height=1cm}]
\tikzstyle{myseq}=[]
\def\fy{0.4}
      \node[myseq] (reads) at (3, 3*\fy) {ACGTTCGTAT, TACAGCTCGA, TCGTAGCTAC, GTTCGTATAT};
      \node[myseq] at (3, 2*\fy) {GCTCGAGTAG , CTACTCGAGC, TATCTATCAA, TCAAACTACA};

%DNA
\node[myseq] at (3,0)      {ACGTTCGTATATCGTAGCTACTCGAGCTGTAGTTTGATAGATAT};
\node[myseq] (mapped) at (3,-1*\fy) {[-------->..................................};
\node[myseq] at (3,-2*\fy) {.....................<--------].............};
\node[myseq] at (3,-3*\fy) {...........[-------->.......................};
\node[myseq] at (3,-4*\fy) {..[-------->................................};
\node[myseq] at (3,-5*\fy) {.................<--------].................};
\node[myseq] at (3,-6*\fy) {.................[-------->.................};
\node[myseq] at (3,-7*\fy) {.................................<--------].};
\node[myseq] at (3,-8*\fy) {...........................<--------].......};
% 0, 7, 11, 2,
% 3, 17, 19, 13
\node[myseq] at (3, -10*\fy) {(0, 10, +), (21, 31, -), (7, 17, +), (2, 12, +)};
\node[myseq] at (3, -11*\fy) {(17, 27, -), (17, 27, +), (33, 43, -), (27, 37, -)};

\node[myseq] at (3, 0)      {ACGTTCGTATATCGTAGCTACTCGAGCTGTAGTTTGATAGATAT};
\node[myseq] (extended) at (3,-13*\fy) {[--------++++++++++++++]....................};
\node[myseq] at (3,-14*\fy) {.......[++++++++++++++--------].............};
\node[myseq] at (3,-15*\fy) {...........[--------++++++++++++++].........};
\node[myseq] at (3,-16*\fy) {..[--------++++++++++++++]..................};
\node[myseq] at (3,-17*\fy) {...[++++++++++++++--------].................};
\node[myseq] at (3,-18*\fy) {.................[--------++++++++++++++]...};
\node[myseq] at (3,-19*\fy) {...................[++++++++++++++--------].};
\node[myseq] at (3,-20*\fy) {.............[++++++++++++++--------].......};

\node[myseq] at (3, -22*\fy) {(0, 24), (7, 31), (7, 31), (2, 26)};
\node[myseq] at (3, -23*\fy) {(3, 27), (17, 41), (19, 43), (13, 37)};


\node[myseq] (pileup) at (3,-25*\fy) {...................-----....................};
\node[myseq] at (3,-26*\fy) {.................---------..................};
\node[myseq] at (3,-27*\fy) {.............--------------.................};
\node[myseq] at (3,-28*\fy) {...........--------------------.............};
\node[myseq] at (3,-29*\fy) {.......----------------------------.........};
\node[myseq] at (3,-30*\fy) {...----------------------------------.......};
\node[myseq] at (3,-31*\fy) {..---------------------------------------...};
\node[myseq] at (3,-32*\fy) {-------------------------------------------.};

\node[myseq] (peak) at (3,-34*\fy) {...................-----....................};

\node[left=of reads] {Reads:};
\node[left=of mapped] {Mapped:};
\node[left=of extended] {Extended:};
\node[left=of pileup] {Pileup:};
\node[left=of peak] {Peak:};
% \node[seq] at (3, -25*\fy) {+.++..+...+.+...+.+...-.--..-...-.-...-.-..};

\end{tikzpicture}
\caption{Illustration of the main MACS2 algorithm. After reads are mapped to the reference genome, all mapped intervals are extended to match the estimated fragment length. A pileup is then created based on the coverage of each position, and positions with a high enough coverage are marked as reads}
%\end{subfigure}
\end{figure}
