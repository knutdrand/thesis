  \centering
\begin{subfigure}{\textwidth}
  \begin{tikzpicture}[decoration={coil},
dna/.style={decorate, thick, decoration={aspect=0, segment length=1.0cm}},
protein/.style={ellipse, draw=white, minimum width=1cm, minimum height=1cm}]
 
%DNA
\draw[dna, decoration={amplitude=.15cm}] (.0,0) -- (11,0);
\draw[dna, decoration={amplitude=-.15cm}] (0,0) -- (11,0);
\node at (0,0.5) {DNA};
 
%Gene
 \node [rounded corners, fill=green!50, thick, inner xsep=30pt] at (8,0) (box){Gene X};
 \node [protein, minimum height=.75cm,fill=blue!30] at (3,0.5) {CTCF};
\node at (3.5, -0.5) {Transcription Factor};
\draw (1.5, -0.2) rectangle (4.5, 1); 
\end{tikzpicture}
\end{subfigure}
\begin{subfigure}{\textwidth}
  \begin{tikzpicture}[decoration={coil},
dna/.style={decorate, thick, decoration={aspect=0, segment length=1.0cm}},
protein/.style={ellipse, draw=white, minimum width=1cm, minimum height=1cm}]
\tikzstyle{seq}=[font=\ttfamily]
 
%DNA
\node[seq] at (3,0) {ACGTTCGTATATCGTAGCTACTCGAGCTGTAGTTTGATAGATAT};
%Gene
\node [protein, minimum height=.75cm,fill=blue!30] at (3,0.5) {CTCF};
\end{tikzpicture}
\caption{Transcription factor binding to DNA, regulating the expression of Gene X}
\end{subfigure}
\begin{subfigure}{\textwidth}
  \begin{tikzpicture}[decoration={coil},
dna/.style={decorate, thick, decoration={aspect=0, segment length=1.0cm}},
protein/.style={ellipse, draw=white, minimum width=1cm, minimum height=1cm}]
seq/.style={font=\ttfamily}
\tikzstyle{seq}=[font=\ttfamily]
\def\fy{0.4}
% DNA
\node[seq] at (3, 0*\fy) (c1) {ACGTTCGTATATCGTAGCTACTCGAGCTGTAGTTTGATAGATAT};
\node[seq] at (3,-1*\fy) (c2) {ACGTTCGTATATCGTAGCTACTCG....................};
\node[seq] at (3,-2*\fy) (c3) {.......TATATCGTAGCTACTCGAGCTGTA.............};
\node[seq] at (3,-3*\fy) (c4) {...........TCGTAGCTACTCGAGCTGTAGTTT.........};
\node[seq] at (3,-4*\fy) (c5) {..GTTCGTATATCGTAGCTACTCGAG..................};
\node[seq] at (3,-5*\fy) {...TTCGTATATCGTAGCTACTCGAGC.................};
\node[seq] at (3,-6*\fy) {.................CTACTCGAGCTGTAGTTTGATAGA...};
\node[seq] at (3,-7*\fy) {...................ACTCGAGCTGTAGTTTGATAGATA.};
\node[seq] at (3,-8*\fy) {.............GTAGCTACTCGAGCTGTAGTTTGA.......};

\node[left=of c1] {Cell1: };
\node[left=of c2] {Cell2: };
\node[left=of c3] {Cell3: };

%Gene
\node [protein, minimum height=.75cm,fill=blue!30] at (3,0.5) {CTCF};
%0, 7, 11, 2, 3, 17, 19, 13
% 0, 2, 3, 7, 11, 13, 17, 19
\end{tikzpicture}
\caption{DNA Fragments obtained from ChIP different cells}
\end{subfigure}
\begin{subfigure}{\textwidth}
  \begin{tikzpicture}[decoration={coil},
dna/.style={decorate, thick, decoration={aspect=0, segment length=1.0cm}},
protein/.style={ellipse, draw=white, minimum width=1cm, minimum height=1cm}]
seq/.style={font=\ttfamily}
\tikzstyle{seq}=[font=\ttfamily]
\def\fy{0.4}
\node [protein, minimum height=.75cm,fill=blue!30] at (3,0.5) {CTCF};

% DNA
\node[seq] at (3, 0*\fy) {ACGTTCGTATATCGTAGCTACTCGAGCTGTAGTTTGATAGATAT};
\node[seq] at (3,-1*\fy) {\textcolor{red}{ACGTTCGTAT}ATCGTAGCTACTCG....................};
\node[seq] at (3,-2*\fy) {.......TATATCGTAGCTAC\textcolor{red}{TCGAGCTGTA}.............};
\node[seq] at (3,-3*\fy) {...........\textcolor{red}{TCGTAGCTAC}TCGAGCTGTAGTTT.........};
\node[seq] at (3,-4*\fy) {..\textcolor{red}{GTTCGTATAT}CGTAGCTACTCGAG..................};
\node[seq] at (3,-5*\fy) {...TTCGTATATCGTAG\textcolor{red}{CTACTCGAGC}.................};
\node[seq] at (3,-6*\fy) {.................\textcolor{red}{CTACTCGAGC}TGTAGTTTGATAGA...};
\node[seq] at (3,-7*\fy) {...................ACTCGAGCTGTAGT\textcolor{red}{TTGATAGATA}.};
\node[seq] at (3,-8*\fy) {.............GTAGCTACTCGAGC\textcolor{red}{TGTAGTTTGA}.......};
% Gene
\draw[->] (3, -8.5*\fy) -- (3, -9.5*\fy);
\node[seq] at (3, -10*\fy) {ACGTTCGTAT, TACAGCTCGA, TCGTAGCTAC, GTTCGTATAT};
\node[seq] at (3, -11*\fy) {GCTCGAGTAG , CTACTCGAGC, TATCTATCAA, TCAAACTACA};
\end{tikzpicture}
\caption{One end of each fragment is sequenced}
\end{subfigure}

%%% Local Variables:
%%% mode: latex
%%% TeX-master: "sequencing.tex"
%%% End:
