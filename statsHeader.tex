\usepackage[T1]{fontenc}
\usepackage[utf8]{inputenc}
\usepackage{amsmath}
\usepackage{amssymb}
\usepackage{listings}
\usepackage{graphicx}
\usepackage{float}
\usepackage{color}
\definecolor{dkgreen}{rgb}{0,0.6,0}
\definecolor{gray}{rgb}{0.5,0.5,0.5}
\definecolor{mauve}{rgb}{0.58,0,0.82}

\lstset{frame=tb,
language=python,
aboveskip=3mm,
belowskip=3mm,
showstringspaces=false,
columns=flexible,
numbers=none,
keywordstyle=\color{blue},
alsoletter={.},
numberstyle=\tiny\color{gray},
commentstyle=\color{dkgreen},
stringstyle=\color{mauve},
breaklines=true,
breakatwhitespace=true,
tabsize=3
}


\renewcommand{\vec}[1]{\mathbf{#1}}
\newcommand{\code}[1]{\lstinline{#1}}
\newcommand{\seq}[1]{\emph{#1}}
\newcommand{\shortestpath}[2]{D(#1, #2)}
\newcommand{\graphcoord}{\Gamma}
\newcommand{\linearcoord}{\epsilon}
\newcommand{\problem}[1]{\subsection*{#1}}
\newcommand{\size}[1]{\left|#1\right|}
\newcommand{\alphabet}{\Gamma}
\newcommand{\set}[1]{\left\{#1\right\}}
\newcommand{\buildset}[2]{\left\{#1 \mid #2\right\}}
\newcommand{\TODO}[1]{\textcolor{red}{(\emph{#1})}}
\newcommand{\est}[2]{\exists #1 \left[#2\right]}
\newcommand{\subproblem}[1]{\subsubsection*{#1)}}
\newcommand{\subsubproblem}[1]{\paragraph*{#1}}
\newcommand{\rscript}[1]{\lstinputlisting[language=R]{#1.r}}
\newcommand{\xni}[1]{\bar{x}_{#1\cdot}}
\newcommand{\Xni}[1]{\bar{X}_{#1\cdot}}
\newcommand{\exfig}[1]{
\begin{figure}[H]
\includegraphics{#1}
\end{figure}
}
\newcommand{\avg}[1]{\bar{#1}}
\newcommand{\med}[1]{med{x}}
\newcommand{\wave}[1]{\tilde{#1}}
\newcommand{\dx}[0]{dx}
\renewcommand{\natural}[0]{\mathbb{N}}
\newcommand{\integers}[0]{\mathbb{Z}}
\newcommand{\intervals}[0]{\mathbb{I}}
\newcommand{\pardiff}[2]{\frac{\partial #1}{\partial #2}}
\newcommand{\normd}[3]{\frac{1}{\sqrt{2\pi #3 }}e^{-(#1 - #2 )^2/2 #3 }}
\newcommand{\cnormd}[2]{\frac{1}{\sqrt{2\pi #2 }}e^{- #1^2/2 #2 }}
\newcommand{\snormd}[1]{\frac{1}{\sqrt{2\pi}}e^{- #1^2/2}}
\newcommand{\invgamma}[3]{\frac{#3^#2}{\Gamma(#2)} #1^{-#2-1}e^{-#3/#1}}
\newcommand{\tinvgamma}[0]{\text{Inv-Gamma}}
\newcommand{\tlogit}[0]{\text{logit}}
\newcommand{\prop}{=}
%%% Local Variables:
%%% mode: latex
%%% TeX-master: t
%%% End:
