\subsection{DNA}
At the heart of molecular biology is \emph{deoxyribonucleic acid}, DNA.
DNA-molecules consist of pairs of four different \emph{nucleotides} linked together to form a double stranded helix.
The four different nucleotides are often represented by the letters A (\emph{adenin}), C (\emph{cytosin}), T (\emph{thymin}) and G (\emph{guanin}), forming an alphabet of nucleotides $\alphabet_{DNA} = \set{A, C, T, G}$.
Each nucleotide can only be paired with one of the others, such that each base-pair in the helix is either (A-T) or (G-C) or opposite. \TODO{chromosomes}

The most important function of DNA is to serve as a template for the synthesis of proteins, which in turn are responsible for a wide range of bio-molecular functions.
This is a two-step process where DNA is first \emph{transcribed} to RNA, molecules similar to DNA but single stranded and with \emph{thymin} replaced by \emph{uracil} (U).
The resulting RNA-sequence can in turn be translated to proteins by mapping triplets of ribonucleotides to amino acids, which are the building blocks of proteins.

Sequences of DNA that are transcribed to RNA that either have a function in themselves, or are in turn translated to proteins, are called \emph{genes}.
Genes constitute only a minor proporotion of human DNA. The remaining DNA can be important due to the biochemical properties of the DNA itself. 

\subsection{Replication and Mutation}
In addition to serving as template for RNA molecules, DNA can also be replicated. This leads to inheritance both on cellular and organismal level. In normal cell division, the chromosomes are replicated such that each of the two resulting cells have a copy of the DNA from the original cell. Furthermore, germ line cells are copied and transfers half of the organisms chromosomes to an offspring. 

However, this DNA-replication is not always accurate.
Errors can occur leading to a copy of the DNA molecule which is not identical with the original.
Most common are the substitution of a single nucleotide, insertion of a small DNA sequence, or deletion of a small subsequence. Such errors lead to difference between cells within an organism, or difference between individuals.

Difference in the DNA-sequence between two individuals can lead to a change in the transcribed RNA sequence and further in the sequence, and thereby the form and function, of the expressed protein.
It can also lead to less drastic changes such as changes in the RNA-structure or the shape of the DNA molecule itself.
In this way genomic variation can determine differences between specimens of the same species and also differences between the species themselves.

\subsection{Epigenomics}
Difference in DNA-sequence can explain phenotypic \TODO{explain geno/pheno} difference between individuals, but fails to explain the difference between the different cells within an organism.
Within an organism, the cells contain mostly the same DNA-sequences, but exhibit vastly different phenotypes. 

The reason for this difference is mostly attributed to differences in expression levels of genes.
Much attention has been devoted in recent years to the mechanisms determining gene expression levels.
Two high-level consortia, ENCODE~\cite{encode} and Roadmap Epigenomics~\cite{roadmap}, have systematically conducted experiments geared to understanding some of these factors, including: 3D-configuration of the chromosomes, accessiblity of the DNA in different regions, transcription factor binding sites, and methylation patterns across the genome.
Among these, this thesis is concentrated on transcription factor binding sites, but for an introduction to epigenomics as a whole see~\ref{epigenomics}.

Transcription factors are proteins that bind to DNA, with an affinity for certain DNA-sequences.
These proteins signal to other proteins that a gene their vicinity should be transcribed, thus affecting expression levels of genes.
As such the precence or absence of transicripion factors binding to a binding site can affect the phenotype of a cell.
Also, since the binding of transcription factors is dependent on the DNA-sequence, a mutation in the DNA-sequence in a binding site can affect the binding of the protein and thus the expression of a nearby gene. 

%%% Local Variables:
%%% mode: latex
%%% TeX-master: "main"
%%% End:
