The alignment methods described in the previous section do not scale well to large sequences.
Aligning a read $Q[:m]$ to a reference genome $R[:N]$ will require computing $mN$ values.
A NGS sequencing experiment typically yields hundreds of thousands of reads of length 30-300 which need to be located in the human reference genome of length 3 billions.
This would need to calculate $10^{16}$ values, which is intractable even on modern computers.

This has led to much progress since the dawn of NGS in developing alignment methods that avoid the complexity of the exact dynamic programming methods.
This has been achieved mainly by two approaches, creating searchable indexes of the reference genome and using heuristics to limit the search space of possible alignments.

The following is a brief description of the developments in this field, with focus on the aligner used in this thesis' projects, BWA-mem~\cite{bwamem}, which uses a common approach called the \emph{seed-and-extend} paradigm.
After this we will look at the methods used for graph alignment with focus on \emph{vg}~\cite{vg}, which is also central in our work.

\subsection{Linear Mapping}
As in section~\ref{sec:pairwise}, we define the mapping problem as finding the interval $i_R$ for wich the edit distance $D(R[i_R], Q)$ is smallest, and the alignment $A(R[i_R], Q)$. We here introduce some of the heuristics used to solve this problem, with focus on the \emph{seed-and-extend} paradigm. The seed-and-extend paradigm involves using an index to find substrings of the query sequence $Q$ which has an exact match in the reference sequence $R$ (called seeds) and using dynamic programming based methods to align the query sequence to a region around the seeds in the reference sequence. We let $EM(Q, R)$ be the set of all exact matches, represented by the tuples $\buildset{(q, r, l)}{Q[q:q+l]=R[r:r+1]}$. Since the dynamic programming methods can still be computationally expensive, the goal is to make the seeds as small a subset of exact matches as possible, while still making sure that the true alignment covers one of the seeds.
 
\subsubsection{Indexing}
An essential tool for quick alignment is to have a searchable index of the reference genome which is capable of returning sets of interval-pairs from the reference and query sequence for which the reference sequence is identical or near identical to the query sequence.
The indexes used by different tools vary, but can mainly be divided into fixed-length (kmer) indexes~\cite{minimap} and variable-length indexes~\cite{bowtie2, bwalong}.
Fixed length indexes typically use hash tables to store the location of a subset of the kmers in the reference sequence.
Variable length indexes usually use a variation of the \emph{full text minute index} (FM-index)~\cite{fm}, described below.
 % The main idea covered here is the seed-and-extend paradigm.
 % This idea consists of finding a set of exact matches to the reference sequence (seeds), and using these as anchors for using dynamic programming based alignment of the query.

\subsubsection{FM-index}
The FM-index uses a succinct representation of the suffix array~\cite{suffixarray} and Burrows-Wheeler transform~\cite{BWT} of the sequence in order to find exact matches of a query string $S$ in $O(\size{S})$ time (see Figure~\ref{fig:FM}).
Thus if a read has one or more exact matches in the reference sequence it can be found directly using the FM-index.
For inexact matches, it can also be used to find an exact alignment of $Q$ and $R$ in approximately $O(\size{R}^{0.628}\size{Q})$ time~\cite{bwtsw, bwalong}.
Some algorithms also use the FM-index to find exact matches for permutations of the query sequence~\cite{bowtie1, bwashort}, but for longer reads the search space gets too big when allowing for indels.
\begin{figure}
  \tikzpicture
  \matrix (ref) [matrix of math nodes, align=right, row sep=0.2em,column sep=0.2em, minimum width=1em, minimum height=1em, nodes={align=right}, ]{
C & T & A & \textcolor{green}{G} & \textcolor{blue}{C} & \textcolor{red}{T} & \textcolor{green}{G} & C & A & \textcolor{red}{T} & \textcolor{green}{G}\\ 
0 & 1 & 2 & 3 & 4 & 5 & 6 & 7 & 8 & 9 & 10\\};
\node[left=0.2 of ref-1-1] {R:};
\draw (ref-1-7)--(ref-1-6) --(ref-1-5);
\draw (ref-1-11)--(ref-1-10);
\matrix (fm) [matrix of math nodes, align=right, row sep=0.2em,column sep=0.2em, minimum width=1em, minimum height=1em, nodes={align=right}, below=of ref]{
% SA & F & & L & $A_i$ & $C_i$ & $G_i$ & $T_i$ \\
- & \$ & CTAGCTGCAT & $G_0$ & 0 & 0 & 0 & 0\\ 
2 & $A_0$ & GCTGCATG\$C & $T_0$ & 0 & 0 & 1 & 0\\ 
8 & $A_1$ & TG\$CTAGCTG & $C_0$ & 0 & 0 & 1 & 1\\ 
7 & $C_0$ & ATG\$CTAGCT & $G_1$ & 0 & 1 & 1 & 1\\ 
0 & $C_1$ & TAGCTGCATG & \$ & 0 & 1 & 2 & 1\\ 
\textbf{4} & $C_2$ & TGCATG\$CTA & $G_2$ & 0 & 1 & 2 & 1\\ 
10 & $G_0$ & \$CTAGCTGCA & $\textcolor{red}{T_1}$ & 0 & 1 & 3 & 1\\ 
6 & $G_1$ & CATG\$CTAGC & $\textcolor{red}{T_2}$ & 0 & 1 & 3 & 2\\ 
3 & $G_2$ & CTGCATG\$CT & $A_0$ & 0 & 1 & 3 & 3\\ 
1 & $T_0$ & AGCTGCATG\$ & $C_1$ & 1 & 1 & 3 & 3\\ 
9 & $T_1$ & G\$CTAGCTGC & $A_1$ & 1 & 2 & 3 & 3\\ 
5 & $T_2$ & GCATG\$CTAG & $\textcolor{blue}{C_2}$ & 2 & 2 & 3 & 3\\};
\node[above=0.2 of fm-1-1] {SA};
\node[above=0.2 of fm-1-2] {F};
\node[above=0.2 of fm-1-4] {L};
\node[above=0.2 of fm-1-7] {OCC};

\matrix (query) [matrix of math nodes, align=right, row sep=0.1em,column sep=0.2em, minimum width=1em, minimum height=0.5em, nodes={align=right}, below=of fm]{
\textcolor{blue}{C} & \textcolor{red}{T} & \textcolor{green}{G}\\};
\node[color=green, draw, dashed, fit=(fm-7-2) (fm-9-2)] {};
\node[color=red, draw, dashed, fit=(fm-11-2) (fm-12-2)] {};
\node[color=blue, draw, dashed, fit=(fm-6-2)] {};
\draw[dashed, color=red] (fm-7-4) -> (fm-11-2);
\draw[dashed, color=red] (fm-8-4) -> (fm-12-2);
\draw[dashed, color=blue] (fm-12-4) -> (fm-6-2);
\node[left=0.2 of query-1-1] {Q:};
% \node[left=of ref] {R:};
% \node[left=of query] {Q:};


%%% Local Variables:
%%% mode: plain-tex
%%% TeX-master: t
%%% End:

  \endtikzpicture
  \caption{Illustration of backward extension using the last-first (LF) property of the FM index.
    The rows represent sorted suffixes of the reference. The SA column holds the indices in the reference sequence for each suffix, the $F$ column holds the first character of each suffix, while the $L$ column holds the preceding character of each suffix.
    Subscripts give the occurrences of each character in each column. I.e. $T_i$ is the $i$th occurrence of $T$ in that column. Finding the string \emph{CTG} is done by (1) starting with the range of all $G$'s in the F column; (2) finding all $T$'s in this range in the $L$ column; (3) mapping by occurrence number those $T$'s to the F column; (4) mapping the $C$'s in the current range in the L column by occurrence number. The end result is in this case a single row which represents position $4$ in the reference sequences.
    The OCC matrix makes it quick to find the range of occurances of a symbol in the $L$ column.
  }
  \label{fig:FM}

\end{figure}

In order to handle indels in longer reads, the main methodology is to use the FM-index to find exact matches between substrings of the query and reference sequence, called seeds, and then aligning the reads using dynamic programing to intervals surrounding the seed matches on the reference sequence~\cite{bowtie2}.
One way of finding seeds is to find Maximal Exact Matches (MEM)~\cite{longmem}.
Maximal Exact Matches are Exact Matches that cannot be extended in either direction, ie. 
\begin{align*}
  MEM(Q, R) = \{(q, r, l) \in EM \mid &(q, r, l+1) \notin EM \wedge \\
  & (q-1, r-1, l+1) \notin EM \}
\end{align*}
BWA-mem furthers this concept to SuperMaximal Exact Matches (SMEM)~\cite{origsmem}.
A SMEM is a MEM where the query interval cannot be extended in either direction and still yields an Exact Match anywhere on the reference~\ref{fig:smem}.
\begin{align*}
  SMEM(Q, R) = \{(q, r, l) \in EM \mid \nexists r^*[&(q, r^*, l+1) \in EM \vee\\
    &(q-1, r^*, l+1) \in EM] \}
\end{align*}
\begin{figure}
  \tikzpicture
  \matrix (seq) [matrix of math nodes, align=right, row sep=0.5em,column sep=0.5em, minimum width=1em, minimum height=1em, nodes={align=right}, ]{C & T & A & G & C & T & G & C & A & T & G\\ 
G & A & T & C & G & A & C & G & T & A & C\\};
\matrix (q) [matrix of math nodes, align=right, row sep=0.5em,column sep=0.5em, minimum width=1em, minimum height=1em, nodes={align=right}, below=of seq]{T & A & C & T & G & A & T & G\\};
\node[color=red, draw, dashed, rounded corners, fit=(q-1-3) (q-1-5)] {};
\node[color=red, draw, dashed, rounded corners, fit=(seq-1-5) (seq-1-7)] {};
\node[color=green, draw, dashed, rounded corners, fit=(q-1-1) (q-1-2)] {};
\node[color=green, draw, dashed, rounded corners, fit=(seq-1-2) (seq-1-3)] {};
\node[color=green, draw, dashed, rounded corners, fit=(seq-2-2) (seq-2-3)] {};
\node[color=blue, draw, dashed, rounded corners, fit=(q-1-6) (q-1-8)] {};
\node[color=blue, draw, dashed, rounded corners, fit=(seq-1-9) (seq-1-11)] {};
\node[color=blue, draw, dashed, rounded corners, fit=(seq-2-8) (seq-2-10)] {};
  \endtikzpicture
  \caption{SMEMs found between query $Q$ and both strands of reference $R$. Note that the MEM $(q, r, l)=(2, 0, 2)$ (CT) is not a SMEM, since it is contained in the SMEM $(q, r, l) = (2, 4, 3)$}
  \label{fig:smem}
\end{figure}

SMEMs are natural to use as seeds as they cover for each subsequence in $Q$ the longest exact match in $R$.
They are however vulnerable for spurious long matches hiding shorter exact matches.
To account for this BWA-mem allows an option to split long SMEMs into shorter MEMs if they are longer than a certain threshold.
Splitting SMEMs like this increases accuracy, since it increases the number of seeds, but can negatively affect performance. 
In order to find SMEMs, an adaption of the FM index is used, the FMD index, where $FMD(R) = FM(R \concat \bar{R})$.

\subsubsection{Prioritizing and Extending}
The seeds found from the index are next used as seeds for dynamic programming based alignment.
$Q$ is then aligned against the left and right side of $(r, r+l)$ for each seed $(q, r, l)$.
Since this is a computationally expensive step, further limiting the set of seeds is advantageous.
BWA-mem does this by \emph{chaining} the seeds.
This is done by grouping approximately collinear, nearby seeds into chains, and removing small chains that overlap with larger chains.
Approximately collinear means that $\abs{(q_1-q_2) -(r_1-r_2)}<w$ for some set threshold $w$.

\subsection{Graph Mapping}
\label{sec:graphmapping}
Mapping to a graph based reference $G_R$ is similar to the linear case, except that instead of finding a linear interval, the goal is to find a graph interval $i_r$ such that $D(G_R[i_r], Q))$ is minimized.
It is however deceptively more complicated. 
Firstly, because the number of subsequences in a sequence graph grows exponentially with the complexity of the graph, and secondly since chaining subsequence-matches is more complicated due to the possible existence of multiple paths between two matches.
\emph{vg}~\cite{vg} has been at the forefront of mapping to a graph reference in company with 7bridges~\cite{7bridges}, showing that it can lead to better mapping accuracy than BWA-mem.
Below is a brief description of the methodology used by \emph{vg}.

\subsubsection{\emph{vg}}
\emph{vg} uses much the same methodology as BWA-mem to align reads.
It uses the GCSA2-index to find SMEMs, uses chaining to filter the SMEMs, and uses a graph adaption of Smith-Waterman to extend the seeds.
\emph{vg} is able to align reads to more complicated graph structures than the simple directed sequence graphs considered in this thesis.
% For simplicity, the descriptions below will be contained to simple graphs, which entails that the GCSA index described is GCSA1\cite{gcsa1} which is only able to index directed sequence graphs.

\subsubsection{GCSA}
The \emph{Generalized compressed suffix array} (GCSA)-index~\cite{gcsa1, gcsa2} is a generalization of the FM-index, where arbitrary-length sequences can be looked up in a sequence graph.
Originally constructed to work on acyclic sequence graphs, version two of GCSA extends the functionality to general variation graphs.
For simplicity we will here focus on acyclic graphs.

GCSA uses the same concept of $LF$ mapping as the FM-index does.
Here the $F$ column holds the label of each node in the graph, sorted by the suffix starting from that node.
The $L$ column holds the labels of the corresponding node's predecessor nodes.
The problem with this setup is that there are many suffixes starting from each node, depending on which path is taken in the graph.
To resolve this problem, the graph needs to be expanded so that for each node, all suffixes starting from that node shares a prefix that is not found from any other node in the graph (see Figure~\ref{fig:gcsa}).

For complicated regions in the graph, this expansion procedure gets too costly, and the GCSA index therefore needs to prune edges in such areas in order to be able to index it.
This means that not all possible sequences in the graph gets indexed. 
Even after pruning, this procedure is costly and makes the GCSA index significantly slower to create, and more memory consuming than the FM-index.

\begin{figure}
  \tikzpicture
  \matrix (seq) [matrix of math nodes, align=right, row sep=0.5em,column sep=0.5em, minimum width=1em, minimum height=1em, nodes={align=right}, ]{G & C & A & C & G\\ 
 &  & T &  & \\};
\draw (seq-1-1) -- (seq-1-2);
\draw (seq-1-1) -- (seq-1-2);
\draw (seq-1-2) -- (seq-1-3);
\draw (seq-1-2) -- (seq-2-3);
\draw (seq-1-3) -- (seq-1-4);
\draw (seq-2-3) -- (seq-1-4);
\draw (seq-1-4) -- (seq-1-5);
\draw (seq-1-4) -- (seq-1-5);

\matrix (seqe) [matrix of math nodes, align=right, row sep=0.5em,column sep=0.5em, minimum width=1em, minimum height=1em, nodes={align=right},right=of seq]{
G & C  & A & C & G\\ 
  & C  & T &   &  \\};
\draw (seqe-1-1) -- (seqe-1-2);
\draw (seqe-1-1) -- (seqe-2-2);
\draw (seqe-1-2) -- (seqe-1-3);
\draw (seqe-2-2) -- (seqe-2-3);
\draw (seqe-1-3) -- (seqe-1-4);
\draw (seqe-2-3) -- (seqe-1-4);
\draw (seqe-1-4) -- (seqe-1-5);
\draw (seqe-1-4) -- (seqe-1-5);
\matrix (gcsa) [matrix of math nodes, align=right, row sep=0.5em,column sep=0.5em, minimum width=1em, minimum height=1em, nodes={align=right},below=of seq]{
SA &  & BWT\\
  & $\$$ & $G_0$ \\
2 & $A_0$ & $C_0$ \\
1 & $C_{0,1}$ & $G_1$ \\
4 & $C_{2}$ & $A_0, \textcolor{red}{T_0}$ \\
6 & $G_{0}$ & $C_1$ \\
0 & $G_{1}$ & $\$$ \\
3 & $T_0$ & $\textcolor{blue}{C_2}$\\
};
\draw[dotted] (gcsa-5-2) -- (gcsa-5-3);
\draw[dotted, color=red] (gcsa-5-3)  -- (gcsa-8-2);
\draw[dotted] (gcsa-8-2) -- (gcsa-8-3);
\draw[dotted, color=blue] (gcsa-8-3) -- (gcsa-5-2);

\matrix (gcsag) [matrix of math nodes, align=right, row sep=0.5em,column sep=0.5em, minimum width=1em, minimum height=1em, nodes={align=right},below=of seqe]{
SA &  & BWT\\
  & $\$$ & $G_0$ \\
2 & $A_0$ & $C_0$ \\
1 & $C_0$ & $G_1$ \\
5 & $C_1$ & $A_0, \textcolor{red}{T_0}$ \\
3  & $C_2$ & $\textcolor{yellow}{G_2}$ \\
7 & $G_{0}$ & $C_1$ \\
\textbf{0} & $G_{1,2}$ & $\$$ \\
4 & $T_0$ & $\textcolor{blue}{C_2}$\\
};
\draw[dotted] (gcsag-5-2) -- (gcsag-5-3);
\draw[dotted, color=red] (gcsag-5-3)  -- (gcsag-9-2);
\draw[dotted] (gcsag-9-2) -- (gcsag-9-3);
\draw[dotted, color=blue] (gcsag-9-3) -- (gcsag-6-2);
\draw[dotted] (gcsag-6-2) -- (gcsag-6-3);
\draw[dotted, color=yellow] (gcsag-6-3) -- (gcsag-8-2);
%\draw[dotted] (gcsa-5-2) -- (gcsa-5-3) -- (gcsa-8-2) -- (gcsa-8-3);\\


\node[above=of seq] {Invalid};
\node[above=of seqe] {Valid};

\node[color=green, draw, dashed, fit=(gcsa-4-2) (gcsa-5-2)] {};
\node[color=red, draw, dashed, fit=(gcsa-8-2)] {};
\node[color=blue, draw, dashed, fit=(gcsa-5-2)] {};

\node[color=green, draw, dashed, fit=(gcsag-4-2) (gcsag-6-2)] {};
\node[color=red, draw, dashed, fit=(gcsag-9-2)] {};
\node[color=blue, draw, dashed, fit=(gcsag-6-2)] {};
\node[color=yellow, draw, dashed, fit=(gcsag-8-2)] {};


\matrix (query) [matrix of math nodes, align=right, row sep=0.1em,column sep=0.2em, minimum width=1em, minimum height=0.5em, nodes={align=right}, below=of gcsa]{
\textcolor{yellow}{G} & \textcolor{blue}{C} & \textcolor{red}{T} & \textcolor{green}{C}\\};
%%% Local Variables:
%%% mode: plain-tex
%%% TeX-master: t
%%% End:

  \endtikzpicture
  \caption{
    \TODO{fix}
    Valid and invalid GCSA index.
    In the first graph, the two edges from $C$ to $A$ and$T$ makes a direct GCSA index impossible due the other $C$ with an edge to a $G$.
    Both $T$ and $A$ need to point to the same row in the index.
    In the valid example, the $C$ node have been duplicated.
    The bifurcation now happens at an earlier point, and the $GC$ prefix now common to all suffices from $G$ is unique so that the the LF mapping is valid.}
    \label{fig:gcsa}
\end{figure}

Since the GCSA index provides the same functionality as an FM-index, it can be used to find SMEMs in much the same manner. These SMEMs are in \emph{vg} used as seeds. 

GCSA version 2 solves the problem in a slightly different way that allows for indexing general variation graphs.
This is based on succinct de-Bruijn graph~\cite{succinctdebruijn} structure which sets a limit to the length of the unique prefixes by allowing for false positive edges to occur in the index.
This means that results from an index lookup need to be validated by traversing the graph. 

\subsubsection{Filtering and Extending}
\emph{vg} employs a similar method as BWA-mem for filtering out seeds: chaining the seeds and removing small chains that overlap bigger chains.
The chaining procedure is however more complicated on a graph, and involves clustering the seeds by position in addition to a Markov Model based method\TODO{cite} to find approximately collinear seeds. 

%%% Local Variables:
%%% mode: latex
%%% TeX-master: "main"
%%% End:
